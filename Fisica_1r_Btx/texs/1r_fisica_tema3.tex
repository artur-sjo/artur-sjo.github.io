
\documentclass{beamer}
\setbeamertemplate{navigation symbols}{}
\usepackage[catalan]{babel}
\usepackage[latin1]{inputenc}
\usepackage{ragged2e}
\usepackage{tikz}
\usepackage{amsmath}
\usepackage{amssymb}
\usepackage[european]{circuitikz}
\usetheme{Warsaw}
\usecolortheme{seagull}
\justifying

\beamertemplatetransparentcovereddynamic
%\beamersetuncovermixins{\opaqueness<1>{25}}{\opaqueness<2->{15}}



\begin{document}
\title{F�sica 1r Batxillerat}
\author{Artur Arroyo}
\date{}


\begin{frame}
\titlepage
\end{frame}

\begin{frame}\frametitle{F�sica 1r Batxillerat}\tableofcontents
%[pausesubsections]
\end{frame}


\section{Circuits el�ctrics simples}
%\begin{frame}\frametitle{Title}
%Each frame should have a title.
%\end{frame}


\subsection{Elements d'un circuit el�ctric}

\begin{frame}\frametitle{Elements d'un circuit el�ctric}
\begin{itemize}
\item
\textbf{Generador:} subministra una difer�ncia de potencial entre dos punts determinats del circuit a fi de comunicar energia als electrons.
\item
\textbf{Conductor:} connecta el�ctricament els diferents elements del circuit. Considerarem que no pateix p�rdues per efecte Joule.
\item
\textbf{Receptors:} reben l'energia el�ctrica que subministra el generador.
\item
\textbf{Interruptors:} dispositius opcionals, permeten obrir o tancar el circuit a voluntat.
\end{itemize}
\end{frame}
\subsection{Associaci� de resist�ncies}
\begin{frame}\frametitle{Associaci� de resist�ncies}
La resist�ncia equivalent de dues $R_{1}$, $R_{2}$ que es troben connectades en s�rie �s la seva suma aritm�tica.
\begin{center}
\begin{tikzpicture}[scale=1]
%\draw[step=1cm,red,very thin] (-3,-3) grid (8,4);

%\draw (2.5,0.5) circle[radius=0.1cm] ;

\draw[color=black, thick]
    (-1,2) to [R, l=$R_1$, o-] (3,2) (3,2) to [R, l=$R_2$, -o] (6,2);

\draw[color=black, thick]
    (-1,-1) to [R,  o-o]  (6,-1)
    (2.5,-0.5) node[]{$R_{eq}=R_{1}+R_{2}$};
\end{tikzpicture}
\end{center}
En aquest cas cal notar que la intensitat que les travessa �s la mateixa i el potencial que cau en cadscuna d'elles �s, en general, diferent.
\end{frame}

\begin{frame}
La resist�ncia equivalent de dues $R_{1}$, $R_{2}$ que es troben connectades en paral�lel �s troba amb la f�rmula $\frac{1}{R_{eq}}{=\frac{1}{R_{1}}+\frac{1}{R_{2}}}$
\begin{center}
\begin{tikzpicture}[scale=1]
%\draw[step=1cm,red,very thin] (-3,-3) grid (8,4);
%\draw (0,0) circle[radius=0.1cm] ;

\draw[color=black, thick] (-2,2) to [short,o-] (0,2) -- (0,3) to [R, l=$R_1$] (5,3) -- (5,1)
to [R, l_=$R_2$] (0,1) -- (0,2)  (5,2) to [short,-o] (7,2);
\draw[color=black, thick] (-1,-1.5) to [R, o-o] (6,-1.5) ;
\draw[color=black, thick] (2.6,-0.8) node[]{$R_{eq}=\frac{R_{1}\cdot R_{2}}{R_{1}+ R_{2}}$};

\end{tikzpicture}
\end{center}
En aquest cas el potencial que cau en cadascuna d'elles �s el mateix i la intensitat que les travessa �s, en general, diferent.
\end{frame}

\begin{frame}\frametitle{Exemple}
Volem calcular la intensitat que circula per la resist�ncia de $20\,\Omega$ del circuit
\begin{center}
\begin{tikzpicture}[scale=1]
%\draw[step=1cm,red,very thin] (-3,-2) grid (8,4);
%\draw (0,0) circle[radius=0.1cm] ;
\draw[color=black, thick] (0,0) to [short,*-] (-2,0) to [R, l=$20\Omega$] (-2,3) -- (0,3)
to [battery1, l=$55V$] (5,3)--(7,3)--(7,0) to [short,*-] (5,0)--(5,1)--(4,1) to [R, l_=$10\Omega$] (1,1)--(0,1)--(0,-1)
--(1,-1) to [R, l=$30\Omega$] (4,-1)--(5,-1)--(5,0);
\end{tikzpicture}
\end{center}
\end{frame}


\begin{frame}
\begin{center}
\begin{tikzpicture}[scale=1]
\draw[color=black, thick] (0,0) --(-2,0) to [R, l=$20\Omega$] (-2,3) --(0,3)
to [battery1, l=$55V$] (5,3) --(7,3) -- (7,0) to [R, l_=$7.5\Omega$] (0,0);
\end{tikzpicture}
\end{center}
\end{frame}


\begin{frame}
\begin{center}
\begin{tikzpicture}[scale=1]
\draw[color=black, thick] (0,0) -- (-2,0)to [R, l=$27.5\Omega$] (-2,3) --(0,3)
to [battery1, l=$55V$] (5,3) --(7,3) -- (7,0) -- (0,0);
\end{tikzpicture}
\end{center}
\end{frame}
\subsection{Derivacions}
\begin{frame}
De forma que tenim $$V=I\cdot R\Longrightarrow I=\frac{55}{27,5}=2\,A$$
Aquesta intensitat �s la que travessa la resist�ncia de $20\Omega$, per� al arribar a la derivaci� es divideix de forma proporcional per les dues branques on hi ha les resist�ncies de $10\Omega$ i $30\Omega$. Per calcular quina intensitat circula per cadascuna farem servir l'esquema seg�ent
\end{frame}

\begin{frame}\frametitle{Derivacions}
\begin{center}
\begin{tikzpicture}[scale=1]
%\draw[step=1cm,red,very thin] (-3,-2) grid (8,4);
%\draw (0,0) circle[radius=0.1cm] ;
\draw[color=black, thick] (0,0) to [short,-*] (-2,0) (7,0) to [short,*-] (5,0)--(5,1)--(4,1) to [R, l_=$R_{1}$] (1,1)--(0,1)--(0,-1)
--(1,-1) to [R, l=$R_{2}$] (4,-1)--(5,-1)--(5,0);
\draw (-1,0.5) node[]{$I$};
\draw (-1,0.3) node[]{$\longrightarrow$};
\draw (1.2,1.5) node[]{$I_{1}$};
\draw (1.2,1.2) node[]{$\longrightarrow$};
\draw (1.2,-1.4) node[]{$I_{2}$};
\draw (1.2,-1.2) node[]{$\longrightarrow$};
\end{tikzpicture}
\end{center}
I es pot demostrar que �s $$I_{1}=I\frac{R_{2}}{R_{1}+R_{2}}\qquad I_{2}=I\frac{R_{1}}{R_{1}+R_{2}}$$
\end{frame}







\end{document}
