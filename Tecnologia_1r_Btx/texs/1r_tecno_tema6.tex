
\documentclass{beamer}
\setbeamertemplate{navigation symbols}{}
\usepackage[catalan]{babel}
\usepackage[latin1]{inputenc}
\usepackage{ragged2e}
\usepackage{tikz}
\usepackage{amsmath}
\usepackage{amssymb}
\usepackage{eurosym}
\usepackage{graphicx}
\usetheme{Warsaw}
\usecolortheme{seahorse}
\justifying
\addtobeamertemplate{block begin}{}{\justifying}
\beamertemplatetransparentcovereddynamic
%\beamersetuncovermixins{\opaqueness<1>{25}}{\opaqueness<2->{15}}



\begin{document}
\title{Tecnologia Industrial 1r Batxillerat}
\author{Artur Arroyo}
\date{}


\begin{frame}
\titlepage
\end{frame}

\begin{frame}\frametitle{Tecnologia Industrial 1r Batxillerat}\tableofcontents
%[pausesubsections]
\end{frame}


\section{Metal�lurgia i sider�rgia}
%\begin{frame}\frametitle{Title}
%Each frame should have a title.
%\end{frame}


\subsection{El proc�s metal�lurgic}
\begin{frame}\frametitle{Obtenci� dels metalls}
\begin{itemize}
\item
\textbf{Mineria:} extracci� del mineral d'un jaciment adequat i la seva preparaci�, separant la part rica en metall d'altres que l'acompanyen.
\item
\textbf{Metal�lurgia:} separaci� del metall dels altres elements amb els quals es troba combinat qu�micament.
\item
\textbf{Ind�stries met�l�liques:} elaboraci� del metall obtingut per a l'obtenci� d'articles �tils.
\end{itemize}
Actualment, tamb� es poden obtenir metalls a partir del reciclatge de productes usats. Aquest �s el cas de l'acer obtingut a partir de les carrosseries dels autom�bils; de l'or, a partir dels circuits integrats dels aparells electr�nics; del coure obtingut dels cables el�ctrics vells; de la plata, a partir de les plaques de radiografies usades, etc.
\end{frame}

\begin{frame}\frametitle{Els minerals}
Els metalls es troben combinats qu�micament amb altres elements. els compostos m�s comuns que formen s�n:
\begin{block}{
\begin{center}
\begin{tabular}{p{0.2\textwidth}|p{0.33\textwidth} |p{0.33\textwidth}}
  Compost & Composici� & Exemple\\
\end{tabular}
\end{center}}
\begin{center}
\begin{tabular}{p{0.2\textwidth}|p{0.33\textwidth} |p{0.33\textwidth}}
  �xids & Metall + oxigen  & Hematites $\rightarrow\,Fe_{2}O_{3}$ \\ \hline
  Sulfurs & Metall + sofre & Galena $\rightarrow\,PbS$ \\ \hline
  Carbonats & Metall + carboni + oxigen & Magnesita $\rightarrow\, MgCO_{3}$ \\
\end{tabular}
\end{center}
\end{block}

\end{frame}

\end{document}
































