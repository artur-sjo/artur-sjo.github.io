\documentclass[a4paper,12pt]{article}
%\pagestyle{empty}

\usepackage[latin1]{inputenc}
\usepackage[catalan]{babel}
\usepackage{amssymb}
\usepackage{amsmath}
\usepackage{multicol}
\usepackage{graphicx}
%\usepackage{helvet}
\topmargin -1.5cm
\begin{document}
\large \textsc{\underline{Exercicis pendents. Oscil�lador harm�nic}}

\normalsize

 \vspace{1cm}
\begin{enumerate}
  \setcounter{enumi}{39}
  \item
   La freq��ncia est� relacionada nom�s amb la massa i la constant el�stica de la molla. La longitud no afecta a la freq��ncia del moviment.
\item
\begin{enumerate}
\item
Tenim $$T_{A}=\frac{2\pi}{\omega_{A}}\qquad T_{B}=\frac{2\pi}{\omega_{B}}$$
dividint les expressions
$$\frac{T_{A}}{T_{B}}=\frac{\sqrt{\frac{m_{A}}{k}}}{\sqrt{\frac{m_{B}}{k}}}$$
$$\frac{T_{A}}{T_{B}}=\sqrt{\frac{\frac{m_{A}}{k}}{\frac{m_{B}}{k}}}=\sqrt{\frac{m_{A}}{m_{B}}}=\sqrt{\frac{2m_{B}}{m_{B}}}=\sqrt{2}$$
\item
no afecta en res. En aquest curs suposem que el per�ode no dep�n de l'amplitud del moviment.
\end{enumerate}
\item
Tenim
$$\omega_{0}=\sqrt{\frac{k}{m_{0}}}\qquad E_{0}=\frac{1}{2}m_{0}(A\omega_{0})^{2}$$
llavors
\begin{enumerate}
\item
$$\omega'=\sqrt{\frac{k}{4m_{0}}}=\frac{1}{2}\omega_{0}$$
\item
$$E'=\frac{1}{2}m'(A\omega')^{2}=\frac{1}{2}4m_{0}\left(A\frac{1}{2}\omega_{0}\right)^{2}=E_{0}$$
\item
$$v'=\pm A\omega'=\pm\frac{1}{2}A\omega_{0}$$
\end{enumerate}
\item
Com que l'energia mec�nica es conserva, tindr� un valor constant, per tant, la gr�fica correcta �s la (a).
\end{enumerate}
\begin{enumerate}
  \setcounter{enumi}{45}
  \item
  \begin{enumerate}
  \item
  En l'equaci�
  $$y(t)=A\cos(\omega t+\varphi)$$
  tenim $$A=0,25\,m$$ $$\omega=2\pi f=2\pi\,rad/s$$  $$\varphi=\pi$$ ja que $$y(t=0)=-A$$
  \item
  $$a_{max}=\pm A\omega^{2}=\pm 0,25\cdot (2\pi)^{2}\,m/s^{2}$$
  la direcci� de l'acceleraci� �s la mateixa que la del moviment, i el sentit �s cap avall al punt m�s alt i cap adalt al punt m�s baix.
  \item
  tenim que
  $$k=m\omega^{2}=3\cdot(2\pi)^{2}\,N/m$$
    \end{enumerate}
\item
\begin{enumerate}
\item
Es t� que $$mg=k\cdot x$$
d'on
$$k=\frac{2\cdot 9,8}{0,12}=163,3\, N/m$$
\item
$$\omega=\sqrt{\frac{k}{m}}=9,04\,rad/s$$
$$T=\frac{2\pi}{\omega}=0,7\,s$$
\item
Suposem que se separa els $10\,cm$ cap avall, llavors
$$y(t)=0,1\cos(9,04t+\pi)$$
\end{enumerate}
\end{enumerate}
\begin{enumerate}
  \setcounter{enumi}{57}
  \item
  La for�a que actua sobre la molla �s $mg$.
  \begin{enumerate}
    \item
   �s
   $$\Delta L=\frac{g}{k}\cdot m$$
    \item
  del pendent de la recta s'obt� $$1,666=\frac{1}{k}$$
   d'on
   $$k=6\,N/m$$
   \end{enumerate}
\item
Sabem que es t� $$F=kx$$
el pendent de la recta �s $\frac{200-0}{0,4-0}=500$ d'on $k=500\,N/m$. El treball que cal fer per estirar la molla $30\,cm$ �s igual a l'energia potencial el�stica emmagatzemada a la molla
$$\mathcal(W)=\frac{1}{2}k\cdot x^{2}=\frac{1}{2}\cdot500\cdot0,3^{2}=22,5\,J$$ 
\end{enumerate}





\end{document} 