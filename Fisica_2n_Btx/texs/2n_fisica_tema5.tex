\documentclass[a4paper,12pt]{article}
%\pagestyle{empty}

\usepackage[latin1]{inputenc}
\usepackage[catalan]{babel}
\usepackage{amssymb}
\usepackage{amsmath}
\usepackage{multicol}
\usepackage{graphicx}
%\usepackage{helvet}
\topmargin -1.5cm
\begin{document}
\large \textsc{\underline{Tema 5 La inducci� electromagn�tica}}

\normalsize

 \vspace{1cm}
\begin{enumerate}
\item
 \textbf{Les experi�ncies de Faraday.}

En el seu afany per produ�r corrent a partir d'un camp el�ctric, Faraday va fer un s�rie d'experiments que es poden resumir en els seg�ents.


\textbf{\underline{1a experi�ncia}}

Faraday va disposar una espira conductora connectada a un galvan�metre. Com que no hi hi ha cap generador al circuit, el galvan�metre no marca pas de corrent. Ara b�, al acostar o allunyar un imant a l'espira, llavors l'agulla del galvan�metre es desvia, assenyalant el pas de corrent per l'espira.

\begin{center}
\includegraphics[scale=0.3]{fara1.jpg}
\end{center}

\textbf{\underline{2a experi�ncia}}

Ara Faraday va usar un solenoide en lloc d'un imant, amb resultats semblants (ja que el solenoide crea un camp magn�tic con l'imant.)
\begin{center}
\includegraphics[scale=0.3]{fara2.jpg}
\end{center}

\textbf{\underline{3a experi�ncia}}

En aquesta experi�ncia, Faraday va col�locar dues espires enfrontades, una connectada a un galvan�metre i l'altra connectada a una bateria amb un iterruptor. Al accionar l'interruptor el galvan�metre marcava pas de corrent per l'espira.
\begin{center}
\includegraphics[scale=0.3]{fara3.jpg}
\end{center}

 Faraday es va adonar que el que hi havia en com� era que sempre que apareixia corrent induit hi havia una variaci� de flux del camp magn�tic.

\item
 \textbf{El flux magn�tic.}
 Donat un camp magn�tic $\vec{B}$ i una superf�cie, es defineix el flux magn�tic $\Phi_{B}$ que travessa la superf�cie, com
$$\Phi_{B}=\vec{B}\cdot\vec{S}=B\cdot S\cos{\theta}$$
on $\vec{S}$ �s un vector normal a la superf�cie i $\theta$, l'angle que formen $\vec{B}$ i $\vec{S}$.
El m�dul de $\vec{S}$ �s l'�rea de la superf�cie.
Les unitats del flux magn�tic s�n tesles per metre quadrat, i s'anomenen \textbf{weber} (\textbf{Wb})
$$ 1\,Wb=1\,T\cdot 1\,m^{2}$$
\item
\textbf{Lleis de la inducci� electromagn�tica}
\begin{itemize}
\item
Llei de Lenz (indica el sentit del corrent induit): el sentit de la intensitat que s'indueix en els experiments de Farady �s tal que crea un camp magn�tic que s'oposa a l'aplicat externament.
\begin{center}
\includegraphics[scale=0.6]{lenz.jpg}
\end{center} 
L'acci� en la figura $a)$ indueix un corrent en l'espira. Aquest corrent crea un camp tal i com es mostra en la figura $b)$. Clarament, aquest camp s'oposa a la variaci� de flux que s'observava en la figura $a)$. Una cosa semblant succeeix en les figures $c)$ i $d)$.
\item
Llei de Faraday (indica el valor de la for�a electromotriu indu�da $\varepsilon$)
$$\varepsilon=-\frac{d\Phi_{B}}{dt}$$
Si no coneixem la depend�ncia de $B$ amb el temps, podem usar
$$\varepsilon=-\frac{\Delta\Phi_{B}}{\Delta t}$$
i gr�cies a la llei d'Ohm podem relacionar la intensitat indu�da $I$ amb aquesta for�a electromotriu $\varepsilon$ i la resist�ncia $R$ de l'espira
$$\varepsilon=I\cdot R$$
\end{itemize}
\end{enumerate}
\end{document} 